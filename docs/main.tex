%%%%%%%%%%%%%%%%%%%%%%%%%%%%%%%%%%%%%%%%%%%%%%%%%%%%%%%%%%%%%%%%%%%%%%
% How to use writeLaTeX: 
%
% You edit the source code here on the left, and the preview on the
% right shows you the result within a few seconds.
%
% Bookmark this page and share the URL with your co-authors. They can
% edit at the same time!
%
% You can upload figures, bibliographies, custom classes and
% styles using the files menu.
%
%%%%%%%%%%%%%%%%%%%%%%%%%%%%%%%%%%%%%%%%%%%%%%%%%%%%%%%%%%%%%%%%%%%%%%

\documentclass[12pt]{article}

\usepackage{sbc-template}

\usepackage{graphicx,url}

\usepackage[brazil]{babel}
\usepackage[utf8]{inputenc}

\usepackage{listings}
\usepackage{xcolor}
\usepackage{float}
\usepackage{amsmath} 
\usepackage{tikz}
\usepackage{subfigure}
\usepackage{float}

% Definição de cores personalizadas
\definecolor{mygreen}{rgb}{0,0.6,0}        % Para comentários
\definecolor{mygray}{rgb}{0.5,0.5,0.5}       % Para números de linha
\definecolor{mymauve}{rgb}{0.58,0,0.82}       % Para strings
\definecolor{myblue}{rgb}{0,0,0.8}            % Para palavras-chave
\definecolor{lightgray}{rgb}{0.95,0.95,0.95}  % Fundo dos códigos

% Configuração padrão para todos os códigos
\lstset{
  backgroundcolor=\color{lightgray},
  basicstyle=\ttfamily\footnotesize,
  breaklines=true,
  captionpos=b,
  numbers=left,
  numberstyle=\tiny\color{mygray},
  frame=single,
  keywordstyle=\color{myblue}\bfseries,
  commentstyle=\color{mygreen}\itshape,
  stringstyle=\color{mymauve},
  tabsize=4,
  showstringspaces=false,
  aboveskip=10pt,
  belowskip=5pt
}
% Estilo para código em C
\lstdefinestyle{CStyle}{
  language=C,
  % Outras opções específicas para C podem ser adicionadas aqui
}

% Definindo uma linguagem para CUDA baseada em C++
\lstdefinelanguage{CUDA}[]{C++}{
  morekeywords={__global__, __device__, __shared__, __host__},
  % Caso necessário, adicione mais palavras-chave ou opções específicas
}

% Estilo para código em Python
\lstdefinestyle{PythonStyle}{
  language=Python,
  morekeywords=[2]{SequentialDiffusionEquation,with,OMPdiffusionEquation,CUDADiffusionEquation},
  % Outras opções específicas para Python podem ser adicionadas aqui
}

\renewcommand{\lstlistingname}{Código}

\sloppy

\title{Engane a IA: Desenvolvendo Perturbações Adversariais Contra Classificadores de DeepFake}

\author{Eduardo Verissimo Faccio, Guilherme Ferreira Lourenço}

\address{Instituto de Ciência e Tecnologia -- Universidade Federal de São Paulo
  (UNIFESP)\\
  São José dos Campos -- SP -- Brasil
  \email{\{verissimo.eduardo,gflourenco\}@unifesp.br}
}

\begin{document}

\maketitle

\begin{resumo}
  Este trabalho explora a vulnerabilidade e a robustez de redes neurais diante de ataques adversariais
  imperceptíveis aos humanos. Utilizando a base de dados “140k Real and Fake Faces”, treinou-se uma rede
  neural convolucional, uma mini UNET e um XGBoost para a classificação de imagens faciais, atingindo alta
  acurácia em condições normais. Em seguida, desenvolveu-se um ataque adversarial baseado em uma arquitetura
  UNET modificada, capaz de gerar perturbações sutis que comprometem significativamente a performance do classificador.
  Os resultados demonstram que, mesmo com perturbações invisíveis a olho nu, a confiabilidade dos sistemas de reconhecimento
  facial pode ser severamente comprometida, ressaltando a importância de desenvolver estratégias para aumentar
  a resiliência dos classificadores.

  \textbf{Palavras-chave}: Ataques adversariais, robustez, redes neurais convolucionais, perturbações imperceptíveis, reconhecimento facial, XGBoost.
\end{resumo}

\section{Introdução}

O avanço das técnicas de aprendizado profundo tem permitido que sistemas de
reconhecimento facial alcancem níveis de acurácia sem precedentes, contribuindo
para uma ampla gama de aplicações, desde segurança pública até interfaces de
usuário baseadas em biometria~\cite{parkhi2015deep}. Entretanto, apesar do
desempenho elevado, diversos estudos demonstraram que redes neurais são
inerentemente vulneráveis a perturbações adversariais - pequenas modificações
imperceptíveis a olho nu que podem levar a classificações
equivocadas~\cite{szegedy2014intriguingpropertiesneuralnetworks,goodfellow2015explainingharnessingadversarialexamples}.

Neste contexto, o presente trabalho investiga a robustez de modelos de
classificação de imagens faciais quando expostos a ataques adversariais sutis.
Utilizando a base de dados ``140k Real and Fake Faces'', foram treinados três
classificadores distintos - uma rede neural convolucional (CNN), uma versão
reduzida de UNET~\cite{} e um classificador baseado em XGBoost~\cite{Chen_2016}
a - que, sob condições normais, alcançaram alta performance. Em seguida,
propõe-se um ataque adversarial que se baseia em uma arquitetura UNET
modificada, cujo objetivo é gerar perturbações imperceptíveis que comprometam a
confiabilidade dos classificadores.

Ao integrar modelos de naturezas distintas na análise, este estudo busca
identificar possíveis diferenças na robustez dos sistemas frente a ataques
adversariais, evidenciando as limitações dos métodos de classificação atuais e
a necessidade de desenvolver estratégias de defesa que aumentem a resiliência
dos sistemas de reconhecimento facial. A abordagem proposta não só contribui
para uma compreensão mais aprofundada das vulnerabilidades dos modelos de
aprendizado profundo, mas também sugere caminhos para a implementação de
mecanismos de defesa que possam mitigar os impactos desses ataques em
aplicações críticas.

\section{Referencial Teórico}
\subsection{Redes Neurais Convolucionais e Reconhecimento Facial}
Redes neurais convolucionais (CNNs) têm se destacado no processamento de
imagens devido à sua capacidade de extrair automaticamente características
hierárquicas, desde padrões simples, como bordas e texturas, até estruturas
complexas, essenciais para o reconhecimento facial. Estudos como o de Parkhi et
al. (2015) demonstram que a utilização de arquiteturas profundas permite
alcançar altos índices de acurácia, tornando as CNNs a escolha preferencial
para aplicações em biometria.

\subsection{Ataques Adversariais}
Apesar do elevado desempenho, as CNNs são vulneráveis a perturbações
adversariais - pequenas modificações intencionais e imperceptíveis na entrada
que podem levar a classificações errôneas. Szegedy et al. (2013) foram
pioneiros ao evidenciar essa fragilidade, enquanto Goodfellow et al. (2014)
apresentaram o Fast Gradient Sign Method (FGSM) para gerar tais perturbações.
Esses ataques exploram a alta dimensionalidade dos dados e a complexidade dos
espaços de decisão dos modelos, comprometendo a confiabilidade dos sistemas de
reconhecimento facial.

\subsection{Estratégias de Geração de Perturbações}
Diversas técnicas de ataque adversarial têm sido propostas, muitas baseadas na
otimização da função de perda do modelo. Entre elas, o FGSM e suas variações
mostram alta eficácia na geração de exemplos adversariais. No presente
trabalho, adota-se uma abordagem inovadora que utiliza uma arquitetura UNET
modificada para criar perturbações sutis. Essa rede adversarial é treinada para
gerar modificações que se mantenham dentro de um intervalo pré-definido (por
exemplo, [epsilon]), garantindo que as alterações sejam imperceptíveis ao olho
humano, mas capazes de degradar significativamente a performance dos
classificadores.

\subsection{Outros Classificadores e Robustez dos Modelos}
Além das CNNs, outros classificadores, como UNET e o XGBoost, têm sido
empregados em problemas de classificação de imagens. O XGBoost, conforme
descrito por Chen Guestrin (2016), utiliza técnicas de boosting que o tornam
robusto e eficiente, especialmente em cenários com dados de alta
dimensionalidade. A comparação entre esses modelos permite avaliar a
variabilidade na robustez frente a ataques adversariais, demonstrando que,
mesmo que a mesma perturbação seja aplicada, a sensibilidade de cada modelo
pode variar, evidenciando a necessidade de estratégias de defesa que aumentem a
resiliência global dos sistemas de reconhecimento facial.

\subsection{Transferibilidade dos Ataques e Estratégias de Defesa}
Um aspecto crítico dos ataques adversariais é a sua capacidade de
transferência, isto é, a perturbação gerada para um modelo pode afetar outros
modelos, mesmo que estes tenham arquiteturas distintas. Essa propriedade
aumenta a ameaça dos ataques em cenários do mundo real, onde o atacante pode
não ter acesso direto ao modelo alvo. Compreender e mitigar essa transferência
é fundamental para o desenvolvimento de mecanismos de defesa robustos que
assegurem a integridade dos sistemas de reconhecimento facial em ambientes
críticos.

\section{Metodologia}

A abordagem adotada neste trabalho visa investigar a robustez de
classificadores de imagens faciais frente a ataques adversariais
imperceptíveis. Para isso, a metodologia foi dividida em três etapas
principais: (i) preparação e treinamento dos modelos, (ii) desenvolvimento do
ataque adversarial e (iii) avaliação dos resultados.

\subsection{Preparação dos Dados e Treinamento dos Modelos}

A base de dados utilizada foi a \textit{``140k Real and Fake Faces''}, composta
por imagens reais e geradas, que foram previamente divididas em conjuntos de
treinamento, validação e teste. A seguir, três modelos foram treinados para a
classificação de imagens faciais:

\begin{itemize}
  \item \textbf{Rede Neural Convolucional (CNN):} Implementada em PyTorch, esta arquitetura é composta por camadas convolucionais, seguidas por camadas de pooling, normalização e camadas totalmente conectadas. Os hiperparâmetros, tais como taxa de aprendizado (\texttt{[learning\_rate]}), número de épocas (\texttt{[epochs]}) e tamanho do batch (\texttt{[batch\_size]}), foram ajustados para maximizar a acurácia no conjunto de validação.

        \begin{figure}[htbp]
          \centering

          \begin{tikzpicture}[scale=0.5, transform shape, node distance=1.5cm, auto]
            % Nodes for CNN
            \node [draw, rectangle, align=center] (inputcnn) {Input\\(128x128x3)};
            \node [draw, rectangle, align=center, below of=inputcnn] (conv1) {Conv 1\\(3$\to$32)};
            \node [draw, rectangle, align=center, below of=conv1] (pool1) {Max Pool};
            \node [draw, rectangle, align=center, below of=pool1] (conv2) {Conv 2\\(32$\to$64)};
            \node [draw, rectangle, align=center, below of=conv2] (pool2) {Max Pool};
            \node [draw, rectangle, align=center, below of=pool2] (conv3) {Conv 3\\(64$\to$128)};
            \node [draw, rectangle, align=center, below of=conv3] (pool3) {Max Pool};
            \node [draw, rectangle, align=center, below of=pool3] (conv4) {Conv 4\\(128$\to$256)};
            \node [draw, rectangle, align=center, below of=conv4] (pool4) {Max Pool};
            \node [draw, rectangle, align=center, below of=pool4] (global) {Global Avg Pool};
            \node [draw, rectangle, align=center, below of=global] (fc1) {FC 1};
            \node [draw, rectangle, align=center, below of=fc1] (fc2) {FC 2};
            \node [draw, rectangle, align=center, below of=fc2] (fc3) {FC 3\\(2 classes)};

            % Arrows for CNN
            \draw [->] (inputcnn) -- (conv1);
            \draw [->] (conv1) -- (pool1);
            \draw [->] (pool1) -- (conv2);
            \draw [->] (conv2) -- (pool2);
            \draw [->] (pool2) -- (conv3);
            \draw [->] (conv3) -- (pool3);
            \draw [->] (pool3) -- (conv4);
            \draw [->] (conv4) -- (pool4);
            \draw [->] (pool4) -- (global);
            \draw [->] (global) -- (fc1);
            \draw [->] (fc1) -- (fc2);
            \draw [->] (fc2) -- (fc3);
          \end{tikzpicture}
          \caption{Diagrama das arquiteturas dos modelos utilizados no trabalho.}
        \end{figure}

  \item \textbf{Mini UNET:} Uma versão reduzida da tradicional arquitetura UNET foi desenvolvida para explorar sua capacidade em extrair características e classificar as imagens, mantendo uma complexidade computacional inferior.

  \item \textbf{XGBoost:} Utilizando o framework XGBoost~\cite{Chen_2016}, foi treinado um classificador baseado em boosting, que se mostrou eficiente para problemas de alta dimensionalidade.
\end{itemize}

Todos os modelos foram treinados sob condições normais, atingindo altos índices
de acurácia no conjunto de teste.

\subsection{Desenvolvimento do Ataque Adversarial}

Após o treinamento dos classificadores, desenvolveu-se um ataque adversarial
com o objetivo de gerar perturbações sutis e imperceptíveis que comprometem a
confiabilidade dos modelos. Para isso, foi utilizada uma arquitetura UNET
modificada, denominada rede adversarial, que recebe como entrada uma imagem
normalizada (em escala \([-1, 1]\)) e gera uma perturbação. Essa perturbação é
então limitada por um mecanismo de \textit{clamp}, de forma a garantir que as
alterações não excedam um intervalo pré-definido \([-\epsilon, \epsilon]\).
Formalmente, a imagem adversarial \( \text{adv\_x} \) é obtida através da
equação:

\[
  \text{adv\_x} = \text{clamp}\Bigl(x + \text{clamp}\bigl(\text{Perturbação}(x), -[\epsilon], [\epsilon]\bigr), -1, 1\Bigr)
\]

onde \( x \) representa a imagem original. Essa abordagem assegura que as
modificações sejam imperceptíveis ao olho humano, mas capazes de induzir erros
nos classificadores.

\begin{figure}[ht]
  \centering
  \begin{tikzpicture}[scale=0.5, transform shape, node distance=1.5cm, auto]
    % Nodes for adversarial network
    \node [draw, rectangle, align=center] (input) {Input\\(128x128x3)};
    \node [draw, rectangle, align=center, below of=input] (down1) {DownBlock 1\\(128$\to$64)};
    \node [draw, rectangle, align=center, below of=down1] (down2) {DownBlock 2\\(64$\to$32)};
    \node [draw, rectangle, align=center, below of=down2] (bottleneck) {Residual Block};
    \node [draw, rectangle, align=center, below of=bottleneck] (up1) {UpBlock 1\\(32$\to$64)};
    \node [draw, rectangle, align=center, below of=up1] (up2) {UpBlock 2\\(64$\to$128)};
    \node [draw, rectangle, align=center, below of=up2] (final) {Final Conv\\(128x128x3)};

    % Arrows for adversarial network
    \draw [->] (input) -- (down1);
    \draw [->] (down1) -- (down2);
    \draw [->] (down2) -- (bottleneck);
    \draw [->] (bottleneck) -- (up1);
    \draw [->] (up1) -- (up2);
    \draw [->] (up2) -- (final);

    % Perturbação: using a tabular to force safe line breaks
    \node [right of=final, xshift=2.5cm, align=center] (clamp) {%
      \begin{tabular}{c}
        Clamp                  \\[2pt]
        $[-\epsilon,\epsilon]$ \\[2pt]
        Perturbação
      \end{tabular}
    };
    \draw [->] (final) -- (clamp);
  \end{tikzpicture}
  \caption{Diagrama das arquiteturas dos modelos utilizados no trabalho.}
\end{figure}

\subsection{Avaliação dos Resultados}

A robustez dos modelos foi avaliada comparando a acurácia obtida sob condições
normais com a acurácia após a aplicação dos ataques adversariais.
Adicionalmente, a técnica GradCAM foi utilizada para visualizar as regiões de
maior atenção dos modelos, permitindo identificar quais áreas (por exemplo,
olhos e lábios) são mais suscetíveis às perturbações.

A avaliação abrangeu os três classificadores de forma integrada, possibilitando
comparar a sensibilidade de cada modelo frente às perturbações geradas pela
rede adversarial. Os resultados quantitativos (como a redução percentual da
acurácia) e as análises qualitativas (obtidas via GradCAM) forneceram uma visão
abrangente sobre as vulnerabilidades dos sistemas de reconhecimento facial.

\section{Conclusão}

\bibliographystyle{sbc}
\bibliography{sbc-template}
\nocite{*}
\end{document}
